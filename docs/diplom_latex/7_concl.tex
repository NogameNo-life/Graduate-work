\chapter*{ЗАКЛЮЧЕНИЕ}
\addcontentsline{toc}{chapter}{ЗАКЛЮЧЕНИЕ}
\label{ch:concl}

В заключение можно сказать, что создание компьютерной модели является важным инструментом для исследования различных физических явлений. В данном дипломном проекте были рассмотрены три симуляции: броуновское движение, распределение молекул по скоростям и распределение частиц в поле тяжести.

Броуновское движение является одним из наиболее изученных явлений в физике. С помощью компьютерной модели удалось продемонстрировать, как частицы движутся в случайном порядке, что позволяет лучше понять механизмы этого явления.

Распределение молекул по скоростям является важным параметром в химии и физике. С помощью компьютерной модели удалось проанализировать, как изменяется распределение молекул при изменении температуры и других параметров.

Распределение частиц в поле тяжести также является важным явлением в физике. С помощью компьютерной модели удалось продемонстрировать, как частицы распределяются в поле тяжести, что позволяет лучше понять механизмы этого явления.

Таким образом, создание компьютерной модели является важным инструментом для исследования различных физических явлений. Разработанные симуляции позволяют лучше понимать механизмы броуновского движения, распределения молекул по скоростям и распределения частиц в поле тяжести.