\chapter{ГЛАВА 2. ИСПОЛЬЗУЕМЫЕ МОДУЛИ}
\label{ch:chapter2}
\section{Компьютерное моделирование и языки программирования}

Одним из наиболее распространенных языков программирования для моделирования является MATLAB. Этот язык используется для решения математических задач, анализа данных и создания графиков. MATLAB имеет большое количество встроенных функций и инструментов для работы с матрицами и векторами, что делает его очень удобным для создания математических моделей.

Другим популярным языком программирования для компьютерного моделирования является Python. Python – это интерпретируемый язык программирования, который используется для разработки программного обеспечения, анализа данных и машинного обучения. Он имеет простой и понятный синтаксис, что делает его доступным для начинающих программистов. Также можно выделить C++, C\#, Java, JavaScript и др. Каждый из этих языков имеет свои преимущества и недостатки в зависимости от задачи, которую нужно решить.

Кроме языков программирования, для компьютерного моделирования используются специализированные программные средства, такие как Simulink, ANSYS, COMSOL и др. Эти программы предоставляют готовые блоки для создания математических моделей и имеют широкий набор инструментов для анализа результатов моделирования.

\section{C++ и классы STL}

C++ $-$ это высокоуровневый язык программирования, который используется для создания приложений с различными целями, включая разработку игр, приложений для мобильных устройств, веб-сайтов и многого другого. Он обладает мощными возможностями, такими как наследование, полиморфизм, шаблоны и многопоточность, что позволяет разработчикам создавать сложные программы с высокой производительностью и надежностью.

STL (Standard Template Library) $-$ это библиотека шаблонов C++, которая предоставляет набор контейнеров, алгоритмов и итераторов для упрощения разработки программ. Она включает в себя множество полезных классов, таких как векторы, списки, множества и карты, которые упрощают хранение и обработку данных. Кроме того, STL также предоставляет мощные алгоритмы для работы с контейнерами, такие как сортировка, поиск и манипуляции со строками. STL позволяет программистам быстро и легко создавать эффективные и надежные приложения.

std::normal\_distribution $-$ это класс библиотеки <random> C++, который представляет нормальное распределение случайных чисел. Он используется для генерации случайных чисел с нормальным распределением, также известным как распределение Гаусса. Нормальное распределение широко используется в статистике и науке о данных для моделирования случайных процессов и анализа данных. Класс std::normal\_distribution позволяет генерировать случайные числа с заданным средним значением и стандартным отклонением, что делает его очень полезным инструментом для различных приложений, включая моделирование финансовых данных, научные исследования и машинное обучение. \cite{CPPREF}

std::random\_device $-$ это класс библиотеки <random> C++, который представляет устройство для генерации случайных чисел. Он используется для получения случайных чисел из физического источника, такого как шум в электрической сети или температурный шум. Это позволяет создавать более случайные последовательности чисел, чем при использовании псевдослучайных алгоритмов. Класс std::random\_device может быть использован в качестве источника случайности для других классов библиотеки <random>, таких как std::mt19937, который представляет генератор псевдослучайных чисел. \cite{CPPREF}

\section{JavaScript и Canvas}

JavaScript $-$ это язык программирования, который широко используется для создания интерактивных веб-страниц. \cite{JSD} Одним из самых популярных приложений JavaScript является создание графики и анимации в вебе с использованием технологии Canvas.

Canvas $-$ это мощный элемент HTML5, который позволяет разработчикам создавать динамическую, интерактивную графику на веб-страницах. С помощью Canvas можно рисовать фигуры, линии, текст и изображения, а также анимировать их в реальном времени. Это делает его идеальным инструментом для создания игр, диаграмм, визуализаций данных и других интерактивных веб-приложений. \cite{CAND}

Одним из ключевых преимуществ использования Canvas с JavaScript является то, что он позволяет быстро и плавно отображать графику, даже на мобильных устройствах. Это происходит потому, что Canvas использует аппаратное ускорение для рендеринга графики, что означает, что он использует GPU (графический процессор) устройства для выполнения вычислений и быстрого рисования графики.

\section{Python и модули для визуализации данных}

Для написания программ используются Python 3.11.2 с пакетами Pygame, Pymunk, random, Matplotlib и Numpy.

Python $-$ высокоуровневый язык программирования общего назначения с динамической строгой типизацией и автоматическим управлением памятью, ориентированный на повышение производительности разработчика, читаемости кода и его качества, а также на обеспечение переносимости написанных на нем программ. Язык является полностью объектно-ориентированным в том плане, что все является объектами. \cite{PYD}

Pygame $-$ набор модулей (библиотек) языка программирования Python, предназначенный для написания компьютерных игр и мультимедиа-приложений. Pygame базируется на мультимедийной библиотеке SDL. \cite{PYGD}

Pymunk $-$ это физический движок для языка программирования Python, который позволяет создавать физические симуляции, такие как игры или моделирование движения объектов. Он предоставляет различные функции для работы с твердыми телами, силами, столкновениями и т.д. \cite{PYMD}

random $-$ это модуль языка программирования Python, который предоставляет функции для генерации случайных чисел. Он может использоваться для создания случайных чисел, выборки случайных элементов из списка, перемешивания списка и т.д. Python random использует различные алгоритмы для генерации чисел, включая алгоритм Mersenne Twister. \cite{PYD}

Matplotlib $-$ это библиотека для языка программирования Python, которая предоставляет широкие возможности для создания различных графиков и визуализации данных. Она позволяет создавать различные типы графиков, включая линейные, столбчатые, круговые, гистограммы, диаграммы рассеяния и многое другое. Matplotlib также предоставляет возможности для настройки внешнего вида графиков, включая цвета, шрифты, размеры и т.д. Библиотека Matplotlib является одной из наиболее популярных библиотек для визуализации данных в Python. \cite{PLOTD}

NumPy (Numerical Python) $-$ это библиотека для языка программирования Python, которая предоставляет поддержку для работы с многомерными массивами и матрицами, а также функции для выполнения математических операций над ними. NumPy является одним из основных инструментов для научных вычислений в Python и используется в таких областях, как машинное обучение, обработка изображений, анализ данных и других. NumPy также предоставляет множество функций для работы с линейной алгеброй, случайными числами, преобразованиями Фурье и другими математическими операциями. \cite{NUMD}
