\documentclass[a4paper,14pt,oneside,openany]{memoir}
\usepackage[utf8]{inputenc}
\usepackage[T1,T2A]{fontenc}
\usepackage[english, russian]{babel}                % Настройки для русского языка
\usepackage{times}
\usepackage{array}
\setlength{\arrayrulewidth}{0.005 mm}
\usepackage{xcolor}
\usepackage{lipsum}
\usepackage{ragged2e}

\usepackage[left=30mm, right=10mm, top=20 mm, bottom=20mm]{geometry}

\usepackage{multirow,makecell}                      % Улучшенное форматирование таблиц
\usepackage{booktabs}                               % Еще один пакет для красивых таблиц
\usepackage{soulutf8}                               % Поддержка переносоустойчивых подчёркиваний и зачёркиваний
\usepackage{icomma}                                 % Запятая в десятичных дробях
\usepackage{hyphenat}                               % Для красивых переносов
\usepackage{textcomp}                               % Поддержка "сложных" печатных символов типа значков иены, копирайта и т.д.
\usepackage[version=4]{mhchem}                      % Красивые химические уравнения
\usepackage{amsmath}                                % Усовершенствование отображения математических выражений 


\pagestyle{plain} % Убираем стандарные для данного класса верхние колонтитулы с заголовком текущей главы, оставляем только номер страницы снизу по центру

%%% Задаем параметры оглавления %%%

\usepackage{indentfirst} % Добавляем отступ к первому абзацу
\linespread{1} % Межстрочный интервал (наиболее близко к вордовскому полуторному) - тут вместо этого используется команда OnehalfSpacing*
\parindent=0.8cm % Абзацный отступ
\renewcommand*{\chapternumberline}[1]{} % Делаем так, чтобы номер главы не печатался
\renewcommand*{\cftchapterfont}{\normalfont\MakeUppercase} % Названия глав обычным шрифтом заглавными буквами
\addto\captionsrussian{\renewcommand\contentsname{Содержание}} % Меняем слово "Оглавление" на "Содержание"
\setrmarg{2.55em plus1fil} % Запрещаем переносы слов в оглавлении

\renewcommand*{\cftchapternumwidth}{1.5em} % Ставим подходящий по размеру разделитель между номером главы и самим заголовком
\setsecnumdepth{subsection} % Номера разделов считать до третьего уровня включительно, т.е. нумеруются только главы, секции, подсекции
\renewcommand*{\chapterheadstart}{} % Переопределяем команду, задающую отступ над заголовком, чтобы отступа не было
\renewcommand*{\printchapternum}{} % То же самое для номера главы
\renewcommand*{\printchaptername}{} % Переопределяем команду, печатающую слово "Глава", чтобы оно не печалось
\renewcommand*{\cftchapterfont}{\normalfont\textbf\MakeUppercase} % Названия глав обычным шрифтом заглавными буквами
\renewcommand*{\cftchapterpagefont}{\normalfont} % Номера страниц обычным шрифтом
\renewcommand*{\cftchapterleader}{\cftdotfill{\cftchapterdotsep}} % Делаем точки стандартной формы (по умолчанию они "жирные")
\renewcommand*{\cftchapterdotsep}{\cftdotsep} % Делаем точки до номера страницы после названий глав
\renewcommand*{\cftdotsep}{1} % Задаем расстояние между точками
\renewcommand*{\chapnumfont}{\normalfont\bfseries} % Меняем стиль шрифта для номера главы: нормальный размер, полужирный
\renewcommand*{\afterchapternum}{\hspace{1em}} % Меняем разделитель между номером главы и названием
\renewcommand*{\printchaptertitle}{\normalfont\bfseries\centering\MakeUppercase} % Меняем стиль написания для заголовка главы: нормальный размер, полужирный, центрированный, заглавными буквами
\setbeforesecskip{10pt} % Задаем отступ перед заголовком секции
\setaftersecskip{10pt} % Ставим такой же отступ после заголовка секции
\setsecheadstyle{\raggedright\normalfont\bfseries} % Меняем стиль написания для заголовка секции: выравнивание по правому краю без переносов, нормальный размер, полужирный
\renewcommand*{\printchapternum}{} 

\maxtocdepth{subsection} % В оглавление попадают только разделы первыхтрех уровней: главы, секции и подсекции

%%% Выравнивание и переносы %%%

\tolerance 1414
\hbadness 1414
\emergencystretch 1.5em                             % В случае проблем регулировать в первую очередь
\hfuzz 0.3pt
\vfuzz \hfuzz
%\dbottom
%\sloppy                                            % Избавляемся от переполнений
\clubpenalty=10000                                  % Запрещаем разрыв страницы после первой строки абзаца
\widowpenalty=10000                                 % Запрещаем разрыв страницы после последней строки абзаца
\brokenpenalty=4991                                 % Ограничение на разрыв страницы, если строка заканчивается переносом


%%% Настраиваем отображение списков %%%

\usepackage{enumitem}                               % Подгружаем пакет для гибкой настройки списков
\makeatletter
    \AddEnumerateCounter{\asbuk}{\russian@alph}     % Объясняем пакету enumitem, как использовать asbuk
\makeatother
\renewcommand{\labelenumii}{\asbuk{enumii}}        % Кириллица для второго уровня нумерации
\renewcommand{\labelenumiii}{\arabic{enumiii}}     % Арабские цифры для третьего уровня нумерации
\setlist{noitemsep, leftmargin=*}                   % Убираем интервалы между пунками одного уровня в списке
\setlist[1]{labelindent=\parindent}                 % Отступ у пунктов списка равен абзацному отступу
\setlist[2]{leftmargin=\parindent}                  % Плюс еще один такой же отступ для следующего уровня
\setlist[3]{leftmargin=\parindent}                  % И еще один для третьего уровня

%%% Счетчики для нумерации объектов %%%

\counterwithout{equation}{chapter}                  % Сквозная нумерация математических выражений по документу
%\counterwithout{table}{chapter}   


%%% Задаем параметры оформления рисунков и таблиц %%%

\usepackage{graphicx, caption, subcaption} % Подгружаем пакеты для работы с графикой и настройки подписей
\graphicspath{{images/}} % Определяем папку с рисунками
\captionsetup[subfigure]{font=small, width=\textwidth, name=Рис., justification=centering} 
\captionsetup[table]{singlelinecheck=false,font=small,width=\textwidth,justification=justified} % Задаем параметры подписей к таблицам: запрещаем переносы, маленький шрифт (в данном случае 12pt), ширина равна ширине текста, выравнивание по ширине
\captiondelim{ --- } % Разделителем между номером рисунка/таблицы и текстом в подписи является длинное тире
\setkeys{Gin}{width=\textwidth} % По умолчанию размер всех добавляемых рисунков будет 
\usepackage[section]{placeins} % Объекты типа float (рисунки/таблицы) не вылезают за границы секциии, в которой они объявлены
% Зачем: Включение номера раздела в номер рисунка. Нумерация рисунков внутри раздела.


\usepackage{caption}
\usepackage{subcaption}
\captionsetup[subfigure]{labelformat=empty} 
%%% Вставляем по очереди все содержательные части документа %%%
\begin{document}

\thispagestyle{empty}

\begin{center}
    Учреждение образования \\
"<Брестский государственный университет имени А. С. Пушкина"> \\
Физико-математический факультет \\
Кафедра общей и теоретической физики \\


    \vspace{20pt}
\end{center}

\begin{flushright}
    \begin{minipage}{0.4\textwidth}
      К защите допустить:\\[0.1em]
      \underline{"<  $\quad$">\hspace*{0.2 cm}} \underline{$\quad \quad \quad$ 2023 г.\hspace*{0.4 cm}} \\ [0.4 em]
      Заведующий кафедрой \\[0.45em]
      \underline{\hspace*{2.8cm}}~Демидчик А.\,В.
    \end{minipage}\\[2.2em]

  \end{flushright}

\vspace{50pt}

  \begin{center}
    \textbf{Дипломная работа} \\
    \vspace{20pt}
  Компьютерное моделирование молекулярно-кинетических процессов

\end{center}
\vfill
    \vspace{20 pt}

\noindent
\begin{tabular}{lp{0.5em}l}
 Выполнила студентка 4 курса гр. КФ-41   && \hspace{-1 cm} \underline{\hspace{3.6cm}} ~Ситковец Я.\,С. \\                          &&   \hspace{- 0.5 cm} \footnotesize{(подпись, дата)} 
 \end{tabular}

\vspace{10 pt}
 \noindent
 \begin{tabular}{lp{4em}l}
   Научный руководитель:   &&  \hspace{- 1.2 cm} Доктор физико-математических наук, \\ \\
                          &&   \hspace{- 1.2 cm} профессор \underline{\hspace{3.4cm}} ~Плетюхов В.\,А. \\
                          &&   \hspace {1.8 cm}\footnotesize{(подпись, дата)}
 \end{tabular}
 \vfill

 \begin{center}
    {\normalsize Брест 2023}
  \end{center}
      % ТИТУЛЬНЫЙ ЛИСТ
\newpage             % Переходим на новую страницу
\setcounter{page}{2} % Начинаем считать номера страниц со второй
%\OnehalfSpacing* % Задаем полуторный интервал текста (в титульнике одинарный, поэтому команда стоит после него)
\tableofcontents*    % Автособираемое оглавление

\chapter*{ВВЕДЕНИЕ}
\addcontentsline{toc}{chapter}{ВВЕДЕНИЕ}
\label{ch:intro}

Компьютерное моделирование $-$ это процесс создания математических моделей реальных систем с использованием компьютерных программ. Оно позволяет исследовать поведение и изменения объектов в различных условиях, а также предсказывать их будущее поведение. Компьютерное моделирование широко используется в науке, инженерии, экономике, медицине и других областях. В настоящее время компьютерное моделирование физических процессов является одним из самых важных инструментов в научных и технических исследованиях. С его помощью можно проводить эксперименты, которые были бы невозможны или слишком дорогостоящими в реальном мире. Кроме того, моделирование позволяет ускорить процесс разработки новых технологий и устройств, а также оптимизировать уже существующие.

Цель данного дипломного проекта заключается в изучении основных методов компьютерного моделирования физических процессов и их применении на практике. Для достижения этой цели будут использованы различные программные средства. В работе будет рассмотрено моделирование различных физических процессов, включая распределение Максвелла-Больцмана, распределение частиц в поле силы тяжести, броуновское движение.
    % ВВЕДЕНИЕ
\input{3_chapter1}


\end{document}

