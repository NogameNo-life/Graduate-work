\chapter*{ВВЕДЕНИЕ}
\addcontentsline{toc}{chapter}{ВВЕДЕНИЕ}
\label{ch:intro}

Компьютерное моделирование $-$ это процесс создания математических моделей реальных систем с использованием компьютерных программ. Оно позволяет исследовать поведение и изменения объектов в различных условиях, а также предсказывать их будущее поведение. Компьютерное моделирование широко используется в науке, инженерии, экономике, медицине и других областях. В настоящее время компьютерное моделирование физических процессов является одним из самых важных инструментов в научных и технических исследованиях. С его помощью можно проводить эксперименты, которые были бы невозможны или слишком дорогостоящими в реальном мире. Кроме того, моделирование позволяет ускорить процесс разработки новых технологий и устройств, а также оптимизировать уже существующие.

Цель данного дипломного проекта заключается в изучении основных методов компьютерного моделирования физических процессов и их применении на практике. Для достижения этой цели будут использованы различные программные средства. В работе будет рассмотрено моделирование различных физических процессов, включая распределение Максвелла-Больцмана, распределение частиц в поле силы тяжести, броуновское движение.
