\chapter*{ОБЩАЯ ХАРАКТЕРИСТИКА РАБОТЫ}
\addcontentsline{toc}{chapter}{ОБЩАЯ ХАРАКТЕРИСТИКА РАБОТЫ}
\label{ch:target}

\noindent \textbf{Цель работы:} \\
\indent Целью данной работы является исследование и моделирование молекулярно-кинетических процессов в различных системах, с использованием компьютерных методов и алгоритмов, анализ полученных результатов и сравнение с теоретическими моделями для понимания физических процессов, лежащих в их основе.  \\


\noindent \textbf{Задачами работы являются:}
\begin{enumerate}
    \item Изучить современные методы компьютерного моделирования.
    \item Разработать модель распределения частиц в поле силы тяжести.
    \item Разработать модель распределения частиц по скоростям.
    \item Разработать модель броуновского движения.
    \item Разработать симуляцию датчика температуры.
\end{enumerate}

\noindent \textbf{Положения, выносимые на защиту}
\begin{enumerate}
    \item Рассмотрены виды и способы моделирования.
    \item Разработаны компьютерные модели и симуляция аналогового устройства.
\end{enumerate}

\noindent \textbf{Личный вклад автора} \\
\indent Дипломнику принадлежит написание программного кода. Научному руководителю принадлежит постановка задачи. \\

\noindent \textbf{Оценка научной значимости результата и практической направленности} \\
\indent \textbf{Научное значение:} \\
\indent Данная работа может быть полезна для разработки новых методов и алгоритмов компьютерного моделирования, которые могут быть применены в других областях науки и техники. \\

\indent \textbf{Практическое значение:} \\
\indent Возможность спользования полученных результатов для оптимизации различных процессов в промышленности, например, для тестирования работы оборудования, а также использование моделей в образовательной сфере.\\

\indent \textbf{Фундаментальный аспект:} \\
\indent Выполнено моделирование молекулярно-кинетических процессов, разработана и внедрена в проект симуляция датчика температуры. \\

\indent \textbf{Прикладной аспект:} \\
\indent Данная работа может быть полезна для разработки новых технологий в области энергетики и окружающей среды. Например, можно использовать результаты моделирования для оптимизации процессов сжигания топлива или для разработки новых материалов для солнечных батарей. Симуляции датчика температуры позволяет определить точность и надежность работы датчика в различных условиях и на разных уровнях температуры. Также данная работа может быть использована для оптимизации процессов охлаждения в промышленности. \\

\noindent \textbf{Структура и объем работы} \\
\indent Дипломная работа состоит из введения, трёх глав, каждая из которых делится на параграфы, заключения и библиографического списка, включающего список использованных источников из 14 наименований. Работа изложена на 37 страниц.



